The first thing I did with the data has been plotting a histogram of the orientation.
In fact, one key assumption that we make in the following is that saccades are internally generated and do not depend on the environment. If the environment does not count, then we expect to see a uniform distribution of the orientation. This is not always the case.

Note that the all the plots in Fig.~\ref{sac_raw_orientation} have the same axis. We note that, even though the data for \Dpseudoobscura is much less than for the other species, it is the cleanest, in the sense that the orientation is very well uniformly distributed.

In the plots for \Dhydei and \Darizonae, instead, we see that the flies tend to orient themselves towards a particular region.