
While it is well defined what we mean by ``amplitude'' of a saccade, its ``duration'' is a bit ambiguous, especially for slower species (\Dpseudoobscura) that have very smooth trajectories. When should one declare the saccade to be concluded?

For the purpose of these graphs, we define the duration as follows:
\[
	\textrm{duration} = \frac{ \textrm{amplitude}  }{ \textrm{top velocity} }
\]
(See Section~\vref{sec:sac_dist-top_velocity} for a discussion of how to compute the top velocity.)

Note that \Dananassae and \Dmelanogaster have exactly the same distribution of duration, but their distributions of amplitude and top velocity are very different.
